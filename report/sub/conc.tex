\section*{Conclusion}

Les séances de Travaux Pratiques du module Systèmes Temps-Réel ont pour objectif d'aboutir à une application fonctionnelle répondant à un cahier des charges donnée.
Cette finalité a pu être atteinte comme le montre la vidéo jointe.
Au-delà de cet objectif tangible, il était question de prendre en main le noyau open source FreeRTOS.
Les questions posées autour de l'ordonnancement et de la gestion mémoire notamment, nous ont amenées à approfondir sa compréhension en étudiant le code source.
\gap
Nous avons également pu renouer avec les notions du temps réel vues l'année passée : tache, sémaphore, file de message, etc.
Les contraintes liées aux systèmes embarqués ont également été abordée.
En effet, une attention particulière à la gestion de la mémoire doit être portée lors du dimensionnement des piles.
Enfin, Cette manipulation sous l'environnement Microchip Studio pour une cible SAMD21 nous a permis d'approfondir l'utilisation avec les outils de debug par exemple.
