\section{Ressource FreeRTOS}

%Une analyse des besoins en éléments de RTOS (tâches, sémaphores, …) requis pour
%l’implantation de l’application présentée par le paragraphe 1 (Application à
%développer)

Le noyau FreeRTOS est un système d'exploitation temps réel (RTOS) ciblé pour les microcontrôleurs et petits processeurs.
Il est distribué sous licence libre MIT.
Le projet est composé d'un noyau et d'un panel de bibliothèques mises à jour régulièrement.
Ces dernières réduisent considérablement le "time to market" en fournissant un panel de programmes destiné aux applications industrielles \cite{web_freeRTOS}.

\subsection{Les tâches}
Une application temps réel basée sur RTOS peut être structurée comme un ensemble de tâches indépendantes. 
Chaque tâche s'exécute dans son propre contexte sans dépendance avec d'autres tâches du système \cite{web_freeRTOS_task}.
Une seule tâche de l'application peut être en cours d'exécution à la fois.
C'est l'ordonnanceur qui décide quelle tâche sélectionner en fonction de nombreux paramètres comme : leur priorité, leur état, les signaux asynchrones d'entré, etc. 
Dans le programme, une tache est une fonction à exécuter qui contient généralement une boucle infinie.
Cette fonction a cependant une particularité, elle ne doit jamais atteindre le retour.
Une tache s'arrête en appelant la procédure \texttt{vTaskDelete( NULL );} alors qu'atteindre la fin de la fonction engendre une exception.
Le programme ci-dessous est donné par FreeRTOS \cite{web_freeRTOS_task} a titre d'exemple et illustre la structure de la fonction d'une tâche.
\begin{lstlisting}[style=CStyle]
    void vATaskFunction( void *pvParameters ) {
        for( ;; )
        {
            -- Task application code here. --
        }
        /* Tasks must not attempt to return from their implementing
        function or otherwise exit. */

        vTaskDelete( NULL );
    }
\end{lstlisting}
Elle respecte un prototype imposé permettant de récupérer des paramètres lors de la création.
La boucle infinie (facultative) vient ensuite.
Cet exemple s'attache à mettre en avant la nécessité de supprimer la tache et de ne jamais sortir de la fonction.

Une fois la fonction d'exécution définie, la tâche doit être créée.
Chaque tâche nécessite de la RAM pour stocker son état et des données comme sa pile. 
Si une tâche est créée à l'aide de \texttt{xTaskCreate()}, la RAM requise est automatiquement allouée. 
Si une tâche est créée en utilisant \texttt{xTaskCreateStatic()} alors la RAM est fournie par le développeur de l'application, elle peut donc être allouée statiquement au moment de la compilation.
La tache est ensuite ajoutée à la liste des taches prêtes à être exécutées.
Le programme ci-dessous est donné par FreeRTOS \cite{web_freeRTOS_createTask} a titre d'exemple et illustre la création d'une tache.
\begin{lstlisting}[style=CStyle]
void Init_function( void ){
    BaseType_t xReturned;
    TaskHandle_t xHandle = NULL;

    /* Create the task, storing the handle. */
    xReturned = xTaskCreate(
                    vATaskFunction,  /* Function that implements the task. */
                    "NAME",          /* Text name for the task. */
                    STACK_SIZE,      /* Stack size in words, not bytes. */
                    ( void * ) 1,    /* Parameter passed into the task. */
                    tskIDLE_PRIORITY,/* Priority at which the task is created. */
                    &xHandle );      /* Used to pass out the created task's handle. */
    if( xReturned == pdPASS ){
        /* The task was created.  Use the task's handle to delete the task. */
        vTaskDelete( xHandle );
    }
}
\end{lstlisting}
L'intérêt principal de cet exemple est l'utilisation de la fonction \texttt{xTaskCreate} et des paramètres associés.
On retrouve premièrement un pointeur vers la fonction précédente \texttt{vATaskFunction}, c'est l'adresse de branchement initiale de la tache.
Vient ensuite le nom, sous forme d'une chaine de caractère, la taille de la pile à allouer, la liste des paramètres à transmettre à la fonction d'exécution, sa priorité et finalement son "identifiant".
Ce dernier permettra par la suite de supprimer une tâche donnée par exemple.
La condition suivante est parfois négligée par les développeurs, mais permet d'assurer que tout s'est bien déroulé dans la création.
Elle est essentielle pour assurer une application robuste.
Une fois la tâche créée est complètement indépendante est sera appelée par l'ordonnanceur.

\subsection{Sémaphores}

Un sémaphore est une variable ou plus généralement un type de données abstrait utilisé pour synchroniser des tâches ou éviter des problèmes de sections critiques. 
Son utilisation passe par les deux fonctions atomiques wait() et signal() respectivement remplacées \texttt{xSemaphoreTake} et \texttt{xSemaphoreGive} dans FreeRTOS.
Ainsi, pour franchir un "wait" il faut nécessairement qu’il soit précédé d’un "signal". Le sémaphore a
été inventé par Edsger Dijkstra. Il se modélise très bien par un réseau de pétri ce qui facilite sa preuve
formelle. Il existe deux types de sémaphore binaire et compteur \cite{sem_wiki}.

Dans un sémaphore binaire l’entier le composant ne peut prendre que deux valeurs 0 ou 1. 
Son fonctionnement est assimilable à un loquet ou le 0 signifie occupé et 1 disponible. 
Il garantit une exclusion mutuelle entre deux fonctions voulant accéder à une ressource critique. 
Un sémaphore binaire peut être utilisée pour une \texttt{synchronisation}, la tache endormie aura alors pour seule information si oui ou non des événements ont eu lieu pendant son sommeil. 
Elle ne saura pas combien. 
Ce n’est pas un défaut, il faut se poser les bonnes questions en fonction du contexte. 
Pour créer un sémaphore binaire, il faut faire appel à la fonction \texttt{xSemaphoreCreateBinary}.
Son prototype est le suivant :
\begin{lstlisting}[style=CStyle]
    SemaphoreHandle_t xSemaphoreCreateBinary( void );
\end{lstlisting}
Aucun paramètre n'est nécessaire.
Il conviendra simplement de vérifier que la fonction ne retourne pas \texttt{NULL}.

Dans l'application temps réelle implémentée par la suite nous utiliserons des sémaphores pour fixer la période d'exécution des tâches.
Il n'est pas nécessaire de vérifier la disponibilité du sémaphore avant de le prendre lors d'une synchronisation.
Le programme ci-dessous donne un exemple d'utilisation pour \texttt{xSemaphoregive}.
\begin{lstlisting}[style=CStyle]
    if( xSemaphore != NULL ) {
        /* Give a semaphore and control if an error occured */
        if( xSemaphoreGive( xSemaphore ) != pdTRUE ) {
           /* pdFALSE if an error occurred */
        }
    }    
\end{lstlisting}
Un simple appel à la fonction \texttt{xSemaphoreGive} suffi à donner le sémaphore.
Les conditions ajoutées contribuent à rendre le programme plus robuste en vérifiant si le sémaphore a bien été créé, et si aucune erreur n'est apparue.

L'attente du sémaphore se fait avec la fonction \texttt{xSemaphoreTake}.
Elle est bloquante et ne peut être franchie que si un jeton (pétri) est disponible.
Sans cela, la tâche passera dans l'état d'attente (wait).
Le programme ci-dessous illustre la prise d'un sémaphore.
\begin{lstlisting}[style=CStyle]

\end{lstlisting}
