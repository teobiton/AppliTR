\section*{Introduction}

\gap

\indent Dans le cadre de l'option Systèmes-Embarqués Temps-Réels en cinquième année d'Électronique et Technologies Numériques, le module ME3 "Système Temps-Réels" est proposé aux étudiants.
Ce module a pour objectif final de développer une application temps-réel simple de mesure de temps de réaction. Pour y parvenir, nous découvrons au fil des séances les informations permettant de conduire au développement de l'application et à l'implantation des éléments requis.
\\\\
\indent Pour mener à bien ce projet, nous utiliserons l'environnement de développement \textit{Microchip Studio} déjà utilisé en quatrième année.
La plus-value est l'importation d'un exécutif Temps-Réel au projet, FreeRTOS, dont nous découvrirons la configuration progressivement.
Enfin, la cible matérielle est un système embarqué composé d'une carte SAMD21XPLAINEDPRO et d'une carte d'extension OLED1XPLAINEDPRO.
Ainsi, le présent rapport présentera dans un premier temps les différents
\\\\
\indent Les codes sources demandés de l'application (fichiers \textit{main.c} et \textit{FreeRTOSConfig.h}) sont disponibles en annexe du présent document.
De plus, pour en prouver le fonctionnement, des photos prises lors du fonctionnement sont ajoutés en rapport (figures \ref{fig:start_app} et \ref{fig:end_app}).
Bien conscients que ces photos ne permettent pas d'attester du fonctionnement de l'application, nous avons également joint une vidéo dans l'archive.