\section{Réponse aux questions}

\subsection{Politique d'ordonnancement} \label{ordonnancement}
Le rôle d'un ordonnanceur est de résoudre le problème de ressources partagées.
Dans le cas d'un RTOS la ressource centrale est le(s) cœur(s) de processeur.
La question étant, sur quels éléments se baser pour déterminer quelle tâche peut y accéder.
Il existe un grand nombre de choix bien documenté dans la littérature.
La politique implémentée dans FreeRTOS est l'ordonnancement préemptif à priorités fixes (Fixed priority pre-emptive scheduling).
Dans le cas de priorité identique un round-robin est appliqué.
Afin de s'approprier le code du noyau, il est demandé d'\textit{identifier la procédure en charge de la politique d’ordonnancement en faisant le lien avec les constantes définies dans le fichier FreeRTOSConfig.h.
Quelle(s) procédure(s) faudrait-il modifier pour changer la politique ?}
\gap

Les procédures liées à l'ordonnancement sont définies dans le fichier \texttt{task.c}.
Parmi elles, on retrouve \texttt{vTaskSwitchContext()}.
Qui, comme son nom l'indique, assure le changement de contexte des taches.
Mais, avant de faire le changement effectif, elle doit identifier la tache la plus prioritaire.
Le bloc de code suivant est extrait de la fonction.
\begin{lstlisting}[style=CStyle]
    /* Select a new task to run using either the generic C or port optimised asm code. */
    taskSELECT_HIGHEST_PRIORITY_TASK();
\end{lstlisting}
Ainsi, pour être plus précis, c'est la macro \texttt{taskSELECT\_HIGHEST\_PRIORITY\_TASK} qui est en charge d'appliquer la politique de préemption à priorité fixe.
Si on souhaite la changer il faut la remplacer.
Dans le cas d'un système temps réel cela n'est pas simple car il faut probablement, pour atteindre de bonnes performances, écrire le code en assembleur.
Il est toutefois, possible d'ajuster la politique avec les paramètres de configuration mis à disposition.
En effet, il est possible de désactiver la préemption dans \texttt{config.h} avec \texttt{configUSE\_PREEMPTION}
Aussi, les priorités dites fixes, car non modifié par le RTOS, sont en fait modifiable durant l'exécution.
Cela aura pour résultat d'influencer la politique.

\subsection{Gestion de la mémoire}
Le noyau du RTOS a besoin de RAM chaque fois qu'une tâche, une file de message, un mutex, un délai logiciel, un sémaphore ou un groupe d'événements est créé. 
La RAM peut être automatiquement allouée dynamiquement à partir du tas, ou elle peut être fournie par le développeur.
La question posée est la suivante :
\textit{Qu’impliquent les 5 politiques de gestion de la mémoire par FreeRTOS ?}
\gap

FreeRTOS offre plusieurs schémas de gestion du tas qui varient en complexité et en fonctionnalités.
Il est possible d'utiliser deux politiques de gestion mémoire simultanément.
Cela permet par exemple, aux piles des tâches et aux autres objets du RTOS d'être placés dans la RAM interne rapide, et aux données de l'application d'être placées dans la RAM externe plus lente \cite{web_freeRTOS_heap}.

\noindent
\textbf{heap 1} \\
C'est le fonctionnement le plus simple, il ne permet pas de libérer de la mémoire une fois allouée.
Malgré cela, il convient à la majorité des applications temps réel.
En effet, quand les contraintes mémoire sont rudes, il est usuel de déclarer les taches et objets RTOS statiquement au démarrage.
Le programme de heap 1 divise une zone mémoire en petits blocs.
La taille globale étant donnée par \texttt{configTOTAL\_HEAP\_SIZE}.

\noindent
\textbf{heap 2} \\
Le second mode propose une fonctionnalité supplémentaire, il permet de libérer de la mémoire, mais ne permet pas la fusion de blocs libres adjacents.
Il peut être utilisé dans le cas où l'application affecte de la mémoire dynamiquement en connaissance des risques.
En effet, la taille des blocs alloués doit être constante sans quoi la mémoire risque de se fragmenter au cours de l'exécution et finir par causer une erreur d'allocation.

\noindent
\textbf{heap 3} \\
Ce mode est bien différent des autres puisqu'il encapsule des fonctions standard \texttt{malloc}() et \texttt{free}().
L'environnement de compilation doit contenir le code de ces deux fonctions.
Les inconvénients à cela sont le non-déterminisme de l'exécution et l'occupation mémoire.
En effet, d'un appel à l'autre le temps d'exéction de \texttt{malloc} et \texttt{free} diffère.

\noindent
\textbf{heap 4} \\
Ce schéma s'apparente au heap 2 avec en plus la fusion des blocs libres adjacents pour éviter la fragmentation.
Cela le rend lui aussi non déterministe.
Il reste cependant bien plus performant que la plupart des implantations de \texttt{malloc}.

\noindent
\textbf{heap 5} \\
Le cinquième est dernier mode de gestion est le plus avancées mis à disposition par FreeRTOS.
Il propose de la possibilité d'étendre le tas (heap) à travers plusieurs zones de mémoire non adjacentes.


\subsection{Fréquence de clignotement}
% La réponse à la question posée en fin du paragraphe 4.3.

\textit{Comment faire varier la fréquence de clignotement de la LED ? (plusieurs possibilités)}
\gap

Il existe plusieurs façons de faire varier la fréquence de clignotement de la LED.
Certaines solutions ont un impact uniquement sur le clignotement de LED dans la tâche "LEDControl" du programme mainBlinkLED0.c donné en annexe \ref{led_blinking}.
D'autres solutions ont un impact sur le fonctionnement général du système.
Elles seront donc classées de la plus simple (ou la plus "propre") à mettre en place à la moins judicieuse pour faire varier la fréquence de clignotement de la LED.
Une première solution consiste à ajouter un compteur dans la tâche "BlinkLED" et à tester sa valeur.
Cela permet de diviser la fréquence de clignotement de la LED.
Une implémentation possible est celle-ci:

\begin{lstlisting}[style=CStyle]
    void vCodeClock(void * pvParameters) {
        uint8_t cpt = 0;
        for( ;; ) {
            vTaskDelay(HORLOGE_TASK_DELAY);
            
            // Frequence divisee par 2
            if (cpt == 1) {
                xSemaphoreGive( xSem_H1);
                cpt = 0;
            } else {
                cpt++;
            }
        }
    }
\end{lstlisting}

Dans le même style, une solution consiste à ajouter un compteur dans la tâche "Clock", diminuant ainsi la fréquence d'envoi du sémaphore à la tâche "LEDControl".
Voici une implémentation possible :

\begin{lstlisting}[style=CStyle]
    void vCodeBlinkLED(void * pvParameters) { 
        /* declaration de variables locales de la tache */ 
        bool Blink = false;
        uint8_t cpt = 0;
        
            for (;;) { 
                /* attente du semaphore d eveil */ 
                xSemaphoreTake(Sem, portMAX_DELAY); 
                /* Cycle H*/ 
                if (cpt == 1) {
                    if (Blink == false) {
                        /* turn LED on. */
                        port_pin_set_output_level(LED_0_PIN, LED_0_ACTIVE);
                        Blink = true;
                    } else {
                        /* turn LED off. */
                        port_pin_set_output_level(LED_0_PIN, !LED_0_ACTIVE);
                        Blink = false;
                    }
                    cpt = 0;
                } else {
                    cpt++;
                }
            } 
    } 
\end{lstlisting}

Une deuxième solution consiste à faire varier la constante \textit{HORLOGE\_TASK\_DELAY}.
Cette constante définit la fréquence de fonctionnement globale de l'ordonnanceur.
Dans une application avec plus de tâches ou si plus de sémaphores étaient envoyées par la tâche "Clock", modifier cette constante aurait un impact sur les fréquences d'appel des tâches.

\begin{lstlisting}[style=CStyle]
    // Declaration de la constante de temps d attente dans la tache "Clock"
    #define HORLOGE_TASK_DELAY 1
\end{lstlisting}

Dans la même idée, mais d'une façon moins appropriée, il est possible de faire varier la fréquence de clignotement de la LED en rajoutant des appels à la fonction \texttt{vTaskDelay}.

\begin{lstlisting}[style=CStyle]
    void vCodeClock(void * pvParameters) {
        for( ;; ) {
            // Deux appels de vTaskDelay divisent par deux la frequence
            vTaskDelay(HORLOGE_TASK_DELAY);
            vTaskDelay(HORLOGE_TASK_DELAY);
            xSemaphoreGive( xSem_H1);
        }
    }
\end{lstlisting}

Les solutions présentées ci-dessus interviennent directement dans l'application.
Une autre possibilité cette fois réside dans la modification de l'OS via le fichier FreeRTOSConfig.h.
Dans ce fichier, il est possible de modifier la fréquence de fonctionnement de l'OS.

\begin{lstlisting}[style=CStyle]
#define configTICK_RATE_HZ		( ( TickType_t ) 1000 )
\end{lstlisting}

Cette ligne définit le nombre de Ticks d'horloge par secondes envoyé par l'OS.
Ici, la valeur associée est 1000, soit une fréquence de 1 kHz et 1000 Ticks/secondes.
Modifier cette valeur permet de faire varier l'horloge envoyée par l'OS aux applications et donc de toute application qui tourne sur l'OS.
C'est une possibilité pour faire varier la fréquence de clignotement de la LED, mais elle est peu adaptée.\\
Enfin, la dernière solution proposée est de modifier directement l'horloge source en provenance du microcontrôleur.
L'horloge de référence du micro peut également être modifiée dans le fichier FreeRTOSConfig.h, à la ligne suivante :

\begin{lstlisting}[style=CStyle]
#define configCPU_CLOCK_HZ		(system_clock_source_get_hz( SYSTEM_CLOCK_SOURCE_OSC8M ))
\end{lstlisting}

Par défaut, c'est un oscillateur à 8 MHz qui est l'horloge de référence du CPU.
Dans le fichier conf\_clocks.h, il est possible d'appliquer des facteurs de division d'horloge prédéfinis.

\begin{lstlisting}[style=CStyle]
#  define CONF_CLOCK_CPU_DIVIDER		SYSTEM_MAIN_CLOCK_DIV_1
/* SYSTEM_CLOCK_SOURCE_OSC8M configuration - Internal 8MHz oscillator */
#  define CONF_CLOCK_OSC8M_PRESCALER		SYSTEM_OSC8M_DIV_1
\end{lstlisting}

Ces facteurs de division (de 1 à 255) divisent l'horloge avant qu'elle ne soit envoyé à l'OS par le CPU.
Il est également possible de changer de source d'horloge en choisissant par exemple une horloge à 32 kHz.

\begin{lstlisting}[style=CStyle]
#define configCPU_CLOCK_HZ	 (system_clock_source_get_hz( SYSTEM_XOSC32K_STARTUP_65536 ))
\end{lstlisting}

Peu importe la méthode choisie pour modifier la fréquence du CPU, c'est une modification très risquée.
En effet, il faut s'assurer au préalable que le taux d'occupation permet une telle modification  diminuer la fréquence du CPU, c'est augmenter le taux d'occupation et ralentir fortement les tâches, notamment dans le cas d'une division par un facteur 1048.
Et même si cela est possible, il faut également s'assurer que l'horloge choisit permet d'assurer le nombre de Ticks/seconde définit pour l'OS.
C'est donc une possibilité pour faire varier la fréquence de clignotement de la LED, mais avec des conséquences non négligeables sur l'application.